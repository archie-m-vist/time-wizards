\documentclass{article}

\usepackage[dvipsnames]{xcolor}
\usepackage{tikz}
\usetikzlibrary{shapes,arrows}
\usepackage{caption} 
\usepackage[letterpaper]{geometry}
\usepackage{pdfpages}
\usepackage[normalem]{ulem}
\usepackage{wrapfig}
\usepackage{hyperref}
\usepackage{tcolorbox}
\usepackage{epigraph}

\newcommand{\twsse}{\emph{Time Wizards: The Sober and Serious Edition}}
\newcommand{\tw}{\emph{Time Wizards}}
\newcommand{\sse}{\emph{Sober and Serious Edition}}
\newcommand{\rfe}{\emph{Revised First Edition}}
\newcommand{\vers}{$0.2.2 \beta$}

\title{\twsse{} Errata and Notes}
\author{Time Wizard Archibald}

\begin{document}
\maketitle

This document is intended to accompany Version \vers{} of \twsse{}.

\section{Differences from \emph{Time Wizards: Revised First Edition}} \label{ssec:verdiff}
This section assumes some familiarity with \emph{Time Wizards: Revised First Edition}. If you're
playing \tw{} for the first time using \sse{} for some reason, feel free to skip to the next
section.

The basics of character creation remain the same as in \rfe{}: each character selects five ``verb
the noun'' phrases from lists provided by the Time Master, which serve as their Time Wizard
powers. Added in \sse{} are two numeric values, which are given values from 1-9 and must sum to
10. These characteristics determine the scale and variety of effects that a Time Wizard's powers
can achieve. For the particulars on characteristics and their uses, see the section on
Determining Characteristics.

In terms of mechanics, the classic \rfe{} of \tw{} is a completely different game from \sse{}.
In particular, the entire dice slap system from classic \tw{} has been removed; while some may
say this removes a characteristic part of what makes a game of \tw{} what it is, it was removed
for several reasons: first and foremost, to make the game more accessible. As a rule, people
(even many fa/tg/uys) tend to avoid physical pain; as such, most any other RPG system would be
easier to pitch to a group than a game of \rfe{} or other slap system \tw{}.

Further, a slap system offers a significant advantage to those with stronger physical
characteristics, such as longer arms or better pain tolerance; the shift to a dice-based system
helps balance things physically, something typically welcomed by a group of the sort of people
who gather around a table of dice and pretend to be wizards.

For the new power mechanics, see the section on Time Moments.

The referenced sections should be sufficient for someone familiar with \emph{Time Wizards:
Revised First Edition} to play \sse{}.

\end{document}